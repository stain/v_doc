%% LaTeX Beamer presentation template (requires beamer package)
%% see http://latex-beamer.sourceforge.net/
%% idea contributed by H. Turgut Uyar
%% template based on a template by Till Tantau
%% this template is still evolving - it might differ in future releases!

\documentclass{beamer}

\mode<presentation>
{
\usetheme{Warsaw}

\setbeamercovered{transparent}
}

\usepackage[english]{babel}
\usepackage[latin1]{inputenc}

% font definitions, try \usepackage{ae} instead of the following
% three lines if you don't like this look
\usepackage{mathptmx}
\usepackage[scaled=.90]{helvet}
\usepackage{courier}


\usepackage[T1]{fontenc}


\title{We tell you what you already would know !}

\subtitle{finding meaning,value and utility in the gigs of data}

% - Use the \inst{?} command only if the authors have different
%   affiliation.
%\author{F.~Author\inst{1} \and S.~Another\inst{2}}
\author{Bart van Leeuwen}

% - Use the \inst command only if there are several affiliations.
% - Keep it simple, no one is interested in your street address.
\institute[netlabs.org]
{
 netlabs.org
}

\date{Developer Workshop 2008}


% This is only inserted into the PDF information catalog. Can be left
% out.
\subject{Talks}



% If you have a file called "university-logo-filename.xxx", where xxx
% is a graphic format that can be processed by latex or pdflatex,
% resp., then you can add a logo as follows:

% \pgfdeclareimage[height=0.5cm]{university-logo}{university-logo-filename}
% \logo{\pgfuseimage{university-logo}}



% Delete this, if you do not want the table of contents to pop up at
% the beginning of each subsection:
\AtBeginSection[]
{
\begin{frame}<beamer>
\frametitle{Outline}
\tableofcontents[currentsection]
\end{frame}
}

% If you wish to uncover everything in a step-wise fashion, uncomment
% the following command:

%\beamerdefaultoverlayspecification{<+->}

\begin{document}

\begin{frame}
\titlepage
\end{frame}

%\begin{frame}
%\frametitle{Outline}
%\tableofcontents
% You might wish to add the option [pausesections]
%\end{frame}


\section{Introduction}

\subsection[Definitions]{Definitions}

\begin{frame}
\frametitle{Definitions}
\begin{definition}[Data]
Anything in a form suitable for use with a computer
\end{definition}
\begin{definition}[Information]
A collection of data which possesses some degree of utility, meaning or value
\end{definition}
\end{frame}

\subsection[Current state]{The information paradox}

\begin{frame}
\frametitle{The information paradox}
\begin{itemize}
  \item Do you really know exactly what information you collected and where you
  stored it ??
\end{itemize}
\end{frame}

\begin{frame}
\frametitle{Storage}

\begin{itemize}
  \item Storage locations
  \begin{itemize}
  	\item Local Computer
  	\item Local Network
  	\item Websites
  	\item Portable Media players
  	\item Mobile phones
  	\item \ldots
  \end{itemize}
  \item Low storage price, no need to
  \begin{itemize}
  	\item Clean up working areas
  	\item Maintain, and index information stored
  \end{itemize}
  
\end{itemize}
\end{frame}

\begin{frame}
\frametitle{The 3 degrees of information}

\begin{itemize}
  \item Value
  \begin{itemize}
    \item Does the information you stored still have some value?
  \end{itemize}
  \item Meaning
  \begin{itemize}
    \item Can you tell from the file what it means to you?
  \end{itemize}
  \item Utility
  \begin{itemize}
    \item What is the use of the information stored?
  \end{itemize}
\end{itemize}

\end{frame}

\begin{frame}
\frametitle{The paradox}

\begin{itemize}
  \item After asking yourself these questions
  \begin{itemize}
    \item The amount of information stored is huge
    \item Meaning, value and utility can only be evaluated by looking at the
    information again
  \end{itemize}
  \item Conclusion
  \begin{itemize}
    \item The vast amount of information, renders it unstructured
    \item Too much ``information'' is data
  \end{itemize}
  
\end{itemize}

\end{frame}

\section{How to get a grip}

\subsection[Describing your information]{Describing your information}

\begin{frame}
\frametitle{Application view}

\begin{itemize}
  \item Look beyond the initial context of your applications 
  \begin{itemize}
    \item Think about storing, and describing contextual information
    \item Investigate open standards for describing information
    \begin{itemize}
      \item Dublin core     
    \end{itemize} 
    \item Use open standards for accessing this information
    \begin{itemize}
      \item WSDL ( Web Service Description Language)
    \end{itemize}
    \item Keep away from possible solutions  
  \end{itemize}
\end{itemize}

\end{frame}

\begin{frame}
\frametitle{User view}

\begin{itemize}
  \item Attach the contextual information to your documents 
  \begin{itemize}
    \item On digital photographs
    \begin{itemize}
      \item Where is it
      \item Who is on there     
    \end{itemize} 
    \item On documents
    \begin{itemize}
      \item What event is the document related to
      \item Who are the reviewers
    \end{itemize}
  \end{itemize}
  \item Keep away from possible solutions
\end{itemize}

\end{frame}
\subsection[Using contextual information]{Using contextual information}

\begin{frame}
\frametitle{Using contextual information}

\begin{itemize}
  \item Define services which are attached to certain contextual data
  \begin{itemize}
    \item Looking up locations with Google
  	\item Storing events in your agenda
  	\item Searching keywords on the web
  	\item Storing locations on your navigation tool
	\end{itemize}
  \item Enterprise adaptation
  \begin{itemize}
  	\item Use web service meta data
  	\item Leverage contextual information added by employees to enrich company
  	knowledge base
\end{itemize} 
\end{itemize}
\end{frame}

\subsection[Deja Vu]{Deja Vu}

\begin{frame}
\frametitle{Is this really new ?}

\begin{itemize}
  \item Extended Attributes
  \begin{itemize}
  	\item Keywords
  	\item Application specific attributes
  \end{itemize}
  \item Web based utilities
  \begin{itemize}
    \item iGoogle
    \item Operator 
   \end{itemize} 
\end{itemize}
\end{frame}

\section{Join the voyage}

\subsection[Departure]{Departure}

\begin{frame}
\frametitle{What is happening now}

\begin{itemize}
  \item Current solutions
  \begin{itemize}
  	\item Web based information or local computer based information
  	\item Don't take moving data into account
   	\item Focused on finding data, not information 
  	\item Context is missing on results
  \end{itemize}
  \item	Almost no crossover between web and desktop
  \end{itemize}
\end{frame}

\subsection[Crossing the border]{Crossing the border}

\begin{frame}
\frametitle{Where do we want to go}

\begin{itemize}
  \item Using the web to find additional information on your local documents
  \item Generate information for local use based on web pages you use
  \item Find relations between data stored on your system
  \item Share contextual information through the enterprise
  \item Find new relations in existing enterprise data
\end{itemize}
\end{frame}

\begin{frame}
\frametitle{What are the directions}

\begin{itemize}
  \item Proof of concept
  \begin{itemize}
  	\item NOM Based
  	\item Contextual cross boundary information collection
  \end{itemize} 
  \item Guerrilla tactics
  \begin{itemize}
  	\item WPS Class replacements
  	\item Windows Shell Extensions
  	\item Gnome Panels 
  \end{itemize} 
\end{itemize}
\end{frame}

\subsection[Roundup]{Roundup}

\begin{frame}
\frametitle{Roundup}

Questions or Remarks ??? 
\end{frame}

\end{document}

%% LaTeX Beamer presentation template (requires beamer package)
%% see http://latex-beamer.sourceforge.net/
%% idea contributed by H. Turgut Uyar
%% template based on a template by Till Tantau
%% this template is still evolving - it might differ in future releases!

\documentclass{beamer}

\mode<presentation>
{
\usetheme{Warsaw}

\setbeamercovered{transparent}
}

\usepackage[english]{babel}
\usepackage[latin1]{inputenc}

% font definitions, try \usepackage{ae} instead of the following
% three lines if you don't like this look
\usepackage{mathptmx}
\usepackage[scaled=.90]{helvet}
\usepackage{courier}


\usepackage[T1]{fontenc}


\title{We tell you what you already know !}

\subtitle{finding meaning in the gigs of data}

% - Use the \inst{?} command only if the authors have different
%   affiliation.
%\author{F.~Author\inst{1} \and S.~Another\inst{2}}
\author{Bart van Leeuwen}

% - Use the \inst command only if there are several affiliations.
% - Keep it simple, no one is interested in your street address.
%\institute[netlabs.og]
%{
%\inst{1}%
%Bart van Leeuwen
%}

\date{Developer Workshop 2008}


% This is only inserted into the PDF information catalog. Can be left
% out.
\subject{Talks}



% If you have a file called "university-logo-filename.xxx", where xxx
% is a graphic format that can be processed by latex or pdflatex,
% resp., then you can add a logo as follows:

% \pgfdeclareimage[height=0.5cm]{university-logo}{university-logo-filename}
% \logo{\pgfuseimage{university-logo}}



% Delete this, if you do not want the table of contents to pop up at
% the beginning of each subsection:
\AtBeginSubsection[]
{
\begin{frame}<beamer>
\frametitle{Outline}
\tableofcontents[currentsection,currentsubsection]
\end{frame}
}

% If you wish to uncover everything in a step-wise fashion, uncomment
% the following command:

%\beamerdefaultoverlayspecification{<+->}

\begin{document}

\begin{frame}
\titlepage
\end{frame}

%\begin{frame}
%\frametitle{Outline}
%\tableofcontents
% You might wish to add the option [pausesections]
%\end{frame}


\section{Introduction}

\subsection[Defenitions]{Defenitions}

\begin{frame}
\frametitle{Defenitions}

\begin{itemize}
  \item Data
  \begin{itemize}
  	\item data is anything in a form suitable for use with a computer
  \end{itemize}
  \item Information
  \begin{itemize}
  	\item A collection of data which possesses some degree of utility,
  	value or meaning
  \end{itemize}
\end{itemize}
\end{frame}

\subsection[Current state]{The information paradox}

\begin{frame}
\frametitle{The information paradox}
\begin{itemize}
  \item Do you really know exactly what information you collected and where you
  stored it ??
\end{itemize}
\end{frame}

\begin{frame}
\frametitle{Storage}

\begin{itemize}
  \item Storage locations
  \begin{itemize}
  	\item Local Computer
  	\item Local Network
  	\item Websites
  	\item Portable Media players
  	\item Mobile phones
  	\item \ldots
  \end{itemize}
  \item Low storage price, no need to
  \begin{itemize}
  	\item Clean up working areas
  	\item Maintain, and index information stored
  \end{itemize}
  
\end{itemize}
\end{frame}

\begin{frame}
\frametitle{The 3 degrees of infomation}

\begin{itemize}
  \item Value
  \begin{itemize}
    \item Does the information you stored stil have some value
  \end{itemize}
  \item Meaning
  \begin{itemize}
    \item Can you tell from the file what it means to you
  \end{itemize}
  \item Utility
  \begin{itemize}
    \item What is the use of the information stored
  \end{itemize}
\end{itemize}

\end{frame}

\begin{frame}
\frametitle{The paradox}

\begin{itemize}
  \item after asking your self these questions
  \begin{itemize}
    \item The amount of information stored is huge
    \item meaning, value and utility can only be evaluated by looking at the
    information again
  \end{itemize}
  \item Conclusion
  \begin{itemize}
    \item The vast amount of information, renders it unstructured.
    \item To much information is data
  \end{itemize}
  
\end{itemize}

\end{frame}

\section{Solutions}

\subsection[Describing your information]{Describing your information}

\begin{frame}
\frametitle{Describing your information}

\begin{itemize}
  \item Look beyond the initial context of your applications 
  \begin{itemize}
    \item Think about storing, and describing contextual information.
    \item Investigate open standards for describing information
    \begin{itemize}
      \item Dublin core     
    \end{itemize} 
    \item Search for openstandards on accessing this information
    \begin{itemize}
      \item WSDL
    \end{itemize}
    \item Don't take posibilities into account  
  \end{itemize}
\end{itemize}

\end{frame}

\end{document}

%% LaTeX Beamer presentation template (requires beamer package)
%% see http://latex-beamer.sourceforge.net/
%% idea contributed by H. Turgut Uyar
%% template based on a template by Till Tantau
%% this template is still evolving - it might differ in future releases!

% Voyager presentation for Warpstock Europe 2007
% Adrian Gschwend

\documentclass{beamer}
% Handout:
%\documentclass[handout]{beamer}

\mode<presentation>
{
\usetheme{Warsaw}
\setbeamercovered{transparent}
}

\usepackage{amsmath,amssymb}
\usepackage[latin1]{inputenc}
\usepackage{times}
\usepackage[T1]{fontenc}

\beamertemplatetransparentcovereddynamic

%\logo{\includegraphics[height=0.5cm]{dws06.png}}

\title[netlabs.org - The Voyager Project]
{netlabs.org - The Voyager Project}

\subtitle
{Where Are We Now?}

\author[Adrian Gschwend]
{Adrian~Gschwend}

\institute[netlabs.org]
{
netlabs.org - Open Source Software
}

\date[3.11.2007]
{Warpstock Europe 2007, Valkenswaard, Netherlands}

\subject{OS/2 and eCS development}
\keywords{OS/2 eCS eComStation Voyager}

% \AtBeginSubsection[]
% {
% \begin{frame}<beamer>
% \frametitle{Overview}
% \tableofcontents[part-1]
% \end{frame}
% }


\begin{document}

\begin{frame}
\titlepage
\end{frame}

\begin{frame}
\frametitle{Outline}
\tableofcontents[hideallsubsections]
\end{frame}

\section{History}

\subsection{The Journey}
\begin{frame}[allowframebreaks=0.6]
\frametitle{The Idea}
The Story so Far\ldots
\begin{itemize}
  \item Long process of thinking about the future for several years
  \item First idea with Kernel of MacOS X in Summer 2004
  \item First presentation of that idea at Developers Workshop 2005 in Dresden
  \item Reconsideration of this idea because it doesn't solve the main problem: Desktop
  \item New idea with OpenGL based Desktop with well known toolkits, developed at SYSTEMS fair in Munich
  \item Talks to various people and first presentation of that idea at
  Warpstock Europe 2005 in Dresden
  \item Presentation of first concept and design studies at Developers
  Workshop 2006 in Biel, Switzerland
  \item License decision during Summer 2006
  \item First 0.1 release of \textit{The Design of Voyager} released to the
  public for Warpstock Canada 2006
  \item Coding from various site, namely Chris Wohlgemuth on NOM
  \item Design studies for GTK+ on OS/2 by Dmitry
  \item Happy Hacking in Wintercamp 2007
  \item No more progress on the DOV document because Adrian was in long holidays\ldots
  \item but, progress on the projects itself, thanks to everyone!
\end{itemize}
\end{frame}


* Status
- - NOM: Alpha/Beta Status
- - Desktop: Proof of Concept
- - API: Current state (Wiki, demo...)
- - What is missing on OS/2 (GTK...)

* Ideas (zeigen was wir uns so �berlegt haben in den letzten Monaten)
- - Scripting Interface
---- wpSetup/wpQuerySetup (class methods for controlling objects)
---- object oriented commandline (MS powershell, but better)
-- OpenLDAP etc
-- Unified Data access (fuse, internt protocols)
-- Configuration Database (write kernel&driver config completely from DB, don't feed DB from config files)(hide original config files completely)
-- Mozilla prism (successor to xulRunner) 


* Contribute
- - API Design (devel only)
- - Implementation Ideas: Wiki, draw your own sketches (sample for a topic
in the presentation)
---- sample sketch: drag&drop of skinining objects (simple sketch !) 
- - (Translation, eventuell zu fr�h)
---- presentation of DSW2007 for Resource Management, framework for system support of several languages (systemwide/per application)
--------- eyecatcher: killer feature: live system-wide switch for language 

* Next steps
- - Porting to other platforms
- - Funding

\end{document}